\documentclass{article}
\usepackage{amsmath, amssymb, verbatim, hyperref}
\author{Patrick Foley and Sam Ventura}
\title{The Effect of Fatigue on Team and Player Performance in Basketball and Hockey}

\begin{document}

\maketitle{}

Advanced analytics have transformed the way athletes in all major sports are evaluated.  In the National Basketball Association (NBA), recent research shows that regularized linear regression models provide predictive and interpretable estimates of individual players’ contributions to offense and defense, controlling for the quality of a player’s teammates and opponents.  However, the best NBA players often can elevate the performance of specific teammates who play well together, indicative of a “chemistry” effect between pairs of players.  We examine this effect statistically by modeling chemistry as a latent variable represented by interaction effects between all pairs of NBA players in the linear regression model.  Using L1 regularization on these interaction effects, we identify which player-pairs have positive or negative chemistry and give interpretable estimates of their joint contribution to offense and defense when they play together as opposed to when they play apart.  We summarize the results by showing the player-pairs with the best and worst chemistry in the NBA from 2008--2013, and by determining, empirically, who had the best “Big Three”.  



\end{document}
